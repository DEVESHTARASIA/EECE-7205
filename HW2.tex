
\documentclass[12pt,letterpaper]{article}
\usepackage{fullpage}
\usepackage[top=2cm, bottom=4.5cm, left=2.5cm, right=2.5cm]{geometry}
\usepackage{amsmath,amsthm,amsfonts,amssymb,amscd}
\usepackage{algpseudocode}
\usepackage{lastpage}
\usepackage{enumerate}
\usepackage{fancyhdr}
\usepackage{mathrsfs}
\usepackage{xcolor}
\usepackage{graphicx}
\usepackage{listings}
\usepackage{hyperref}

\hypersetup{%
  colorlinks=true,
  linkcolor=blue,
  linkbordercolor={0 0 1}
}
 
\renewcommand\lstlistingname{Algorithm}
\renewcommand\lstlistlistingname{Algorithms}
\def\lstlistingautorefname{Alg.}

\lstdefinestyle{Python}{
    language        = Python,
    frame           = lines, 
    basicstyle      = \footnotesize,
    keywordstyle    = \color{blue},
    stringstyle     = \color{green},
    commentstyle    = \color{red}\ttfamily
}

\setlength{\parindent}{0.0in}
\setlength{\parskip}{0.05in}

% Edit these as appropriate
\newcommand\course{EECE 7205}
\newcommand\CRN{CRN: 20635}
\newcommand\hwnumber{2}                  % <-- homework number
\newcommand\Name{Devesh Tarasia}
\newcommand\Email{tarasia.dev@northeastern.edu}           % <-- NetID of person #2 (Comment this line out for problem sets)

\pagestyle{fancyplain}
\headheight 35pt
%\lhead{\Name}
\lhead{\Name\\\Email}                 % <-- Comment this line out for problem sets (make sure you are person #1)
\chead{\textbf{\Large Homework \hwnumber}}
\rhead{\course \\ \CRN}
\lfoot{}
\cfoot{}
\rfoot{\small\thepage}
\headsep 1.5em

\begin{document}

\section*{Problem 1}

  \textbf{For groups of size 7},\\
  $\lceil{n/7}\rceil$ groups in this case. Considering $x$ as the median-of-medians, 
  we find atleast half of the groups have 4 or more elements greater than x. The groups that do not have elements larger than 
  $x$ are the group with less than 7 elements (if $n \mod 7 \neq 0$) and the group which contains $x$ 
  (as only 3 elements can be more than $x$).

  The number of elements in each side of partition is atleast is $4*((1/2)*\lceil n/7\rceil-2)\geq 2*(n/7)-8$\\
  So atmost, the elements in each side of partition is $n-(2*(n/7)-8)=5*n/7+8$.\\
  So the recurrence relationship becomes:
  $$T(n) = T(n/7)+T(5n/7+8)+\Theta (n) \; \; \; \forall \; \; n\geq n_o$$ 

  We can now show by substitution that, $T(n) \in O(n)$ or $T(n) \leq cn$.
  $$T(n) \leq c\left\lceil n/7\right\rceil + c(5n/7+8) + \Theta (n)$$
  $$T(n) \leq cn/7 + c + 5cn/7 +8c + \Theta (n)$$
  $$= cn - (cn/7 -9c -O(n)) \leq cn$$

  Since we can pick $c$ and $n_0$ such that, $cn/7 \geq 9c+O(n)$  $\forall$ $n \geq n_0$. 
  The $c$ has to be choosen with consideration of the $\Theta(n)$ term as well. In any case, the larger of the two values (from $O(n)$ and $cn/7 > 9c$)
  can be selected and the condition will still hold. \\
  
  Since, $T(n) = T(n/7)+T(5n/7+8)+\Theta (n) >  n\; \; \; \forall \; n$,\\
  we can easily say that $T(n) \in \Omega (n)$
  
  Thus $T(n) \in \Theta (n)$
  
  \\
  \textbf{For groups of size 3,}\\
  As in the case of groups of size 7 or 5, we find at least half of the groups have 2 or more elements greater than $x$.\\
  So, number of elements greater than $x$ is $2(1/2*\lceil n/3\rceil-2) \geq n/3-4$. So, at most the elements on either side is $n-(n/3-4)=2n/3+4$.\\
  So, the recurrence relation becomes, 
  $$T(n) = T(n/3)+T(2n/3+4)+\Theta (n) \; \; \; \forall \; \; n\geq n_o$$
  
  Hypothesis, $T(n) \in O(n)$ or $T(n) \leq cn$
  
  Trying by substitution,
  $$T(n) \leq c\lceil n/3\rceil + c(2n/3+4) + \Theta (n)$$
  $$T(n) \leq cn/3 + c + 2cn/3 +4c +\Theta (n)$$
  $$= cn + (5c+O(n)) \geq cn \;\;\; \forall \; n$$
  
  So, $T(n) \not\in O(n)$
  
  For the lower bound it can be said that $T(n) \in \Omega (n)$ from the recurrence equation.\\
  But for a tighter bound, $(n\log n)$ is chosen.\\
  Hypothesis: $T(n) \in \Omega (n\log n)$
  By substituition,
  $$T(n) \geq c\lceil n/3*\log {n/3}\rceil + c(2n/3*\log {2n/3}+4) + \Theta (n)$$
  $$T(n) \geq cn/3 + 2cn/3*(1+\log n - \log 3) +4c +\Theta (n)$$
  $$= 2c/3*n\log n + n - 2cn/3*\log 3+ 4  + \Theta (n) \geq c(n\log n) \;\;\; \forall \; n > n_0$$
  An appropriate $n_0$ and $c$ can be selected to satisfy the above case.\\
  Thus $T(n) \in \Omega (n\log n)$
  
\section*{Problem 2}
  The median of the combined two arrays is the median of the two sorted arrays combined into an sorted array. 
  We can have two cases,\\
  Either largest element in $A$ is smaller than smallest element in $B$ or vice versa, in that case the largest element in the smaller array is the median of the two arrays. (One array is appended to another)\\
  The other case is that the arrays merge in a different fashion than being appended. In this case, we can start the search for the median of the two arrays, $x$ by checking the current medians of $A$ and $B$, $m_A$ and $m_B$ respectively.\\
  There are three cases to be considered:\\
  If the $m_A = m_A$, then $x = m_A = m_B$\\
  If $m_A > m_B$, then $x$ is present in $A[0\; ...\;n/2]$ or $B[n/2\; ...\; n]$\\
  or vice versa,\\
  If $m_B > m_A$, similarly, $x$ is present in $B[0\; ...\;n/2]$ or $A[n/2\; ...\; n]$\\
  Using a divide and conquer approach, we can find the medians of the two  smaller arrays and apply the same conditional checks to reach to $x$.\\
  \textit{Edge case:} For arrays of size two elements, we can start apply the merging procedure of the mergesort to a new array $C$ and $x=C[1]$. It will take a fixed time since size is known.
  
  The recursion in case the arrays are merged in some other fashion other being appended is,
  $$T(n) = T(n/2) + O(1)$$
  Since the array size is halved at each iteration.
  
  
  By applying Master's theorem, ($a=1$,$b=2$ & $\log_b a = 1$ $\epsilon = 1$) 
  $$T(n) = \Theta (\log n)$$\\
  
  
  
\section*{Problem 3}
  In this problem, we need to find the $k^{th}$ smallest element and all elements smaller than it.\\
  
  Since, we have been given $O(n)$ time to do pre-processing, 
  
  We use a modified median-of-median method to find the $k^{th}$ element. This takes $O(n)$ time 
  
  After we find the $k^{th}$ we use quicksort's partition algorithm to partition the array around the $k^{th}$ element. This takes $\Theta(n)$ which means $T(n) \in O(n)$, where $T(n)$ is the running time of partition algorithm.
  
  Thus total pre-processing time: $O(n)+O(n)=O(n)$
  
  For printing out the elements, we need to print out the elements till $A[k-1]$, since we do not need the first $k$ smallest elements in a particular order, the running time is $O(k)$
  
  
  %\textbf{Steps:}
  
  
  
  
%\begin{enumerate}
%  \item
%   Problem 1 part 1 answer here.
%  \item
%    Problem 1 part 2 answer here.
%
%    Here is an example typesetting mathematics in \LaTeX
%\begin{equation*}
%    X(m,n) = \left\{\begin{array}{lr}
%        x(n), & \text{for } 0\leq n\leq 1\\
%        \frac{x(n-1)}{2}, & \text{for } 0\leq n\leq 1\\
%        \log_2 \left\lceil n \right\rceil \qquad & \text{for } 0\leq n\leq 1
%        \end{array}\right\} = xy
%\end{equation*}
%
%    \item Problem 1 part 3 answer here.
%
%    Here is an example of how you can typeset algorithms.
%    There are many packages to do this in \LaTeX.
%     
%    \lstset{caption={Caption for code}}
%    \lstset{label={lst:alg1}}
%     \begin{lstlisting}[style = Python]
%    from package import Class # Mesh required for..
%    
%    cinstance = Class.from_obj('class.obj')
%    cinstance.go()
%    \end{lstlisting}
%     
%  \item Problem 1 part 4 answer here.
%
%    Here is an example of how you can insert a figure.
%    \begin{figure}[!h]
%    \centering
%    %\includegraphics[width=0.3\linewidth]{heidi.jpg}
%    \caption{Heidi attacked by a string.}
%    \end{figure}
%\end{enumerate}


%\section*{Problem 2}
% Rest of the work...

\end{document}
